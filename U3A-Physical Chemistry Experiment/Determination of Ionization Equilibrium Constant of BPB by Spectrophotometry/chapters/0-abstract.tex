% !TeX root = ../main.tex

\ustcsetup{
  keywords = {
    物理化学实验, 分光光度法, 溴酚蓝, 电离平衡常数
  },
  keywords* = {
    Physical chemistry experiment, Spectrophotometer, Bromophenol blue, Ionization equilibrium constant
  },
}

\begin{abstract}
  波长为$\lambda$的单色光通过任何均匀而透明的介质时,由于物质对光的吸收作用而使透射光的强度($I$)比入射光的强度($I_0$)要弱,其减弱的程度与所用的波长($\lambda$)有关.~
  又因分子结构不相同的物质,对光的吸收有选择性,因此不同的物质在吸收光谱上所出现的吸收峰的位置及其形状,
  以及在某一波长范围内的吸收峰的数目和峰高都与物质的特性有关.~
  分光光度法是根据物质对光的选择性吸收的特性而建立的,这一特性不仅是研究物质内部结构的基础,也是定性分性、定量分析的基础.~
  本实验借助分光光度法,通过测量pH与吸光度并作图,求得了溴酚蓝(bromophenol blue, BPB)在不同pH下的电离平衡常数.~
\end{abstract}

\begin{abstract*}
  When monochromatic light with a wavelength of $ \lambda $passes through any uniform and transparent medium, the intensity of the transmitted light ($I $) is weaker than that of the incident light ($I_0 $) due to the absorption of light by the substance, and the degree of its weakening depends on the wavelength used ($ \lambda $).
  Due to the selective absorption of light by substances with different molecular structures, the positions and shapes of the absorption peaks that appear in the absorption spectra of different substances are different,
  The number and height of absorption peaks within a certain wavelength range are related to the characteristics of the substance.
  Spectrophotometry is established based on the selective absorption of light by substances, which is not only the basis for studying the internal structure of substances, but also the basis for qualitative and quantitative analysis.
  This experiment used spectrophotometry to determine the ionization equilibrium constants of bromophenol blue (BPB) at different pH values by measuring pH and absorbance and plotting them.
\end{abstract*}
