% !TeX root = ../main.tex

\chapter{序言}

\section{实验原理\cite{huaxe23}}
根据贝尔-郎比定律,溶液对于单色光的吸收,遵守下列关系式
\begin{equation}\label{1.1}
  D=\lg\frac{I_0}{I}=K\cdot l\cdot C
\end{equation}
其中$D$为消光度(或光密度);$\frac{I}{I_0}$为透光率,$K$为摩尔消光系数,
它是溶液的特性常数;$l$为被测溶液的厚度,即吸收槽的长度;
$C$为溶液浓度.~

从式(\ref{1.1})可以看出,对于固定长度的吸收槽,在对应最大吸收峰的波长($\lambda$)下,测定不同浓度$C$的消光,就可以作出线性的$D-C$线,这就是定量分析的基础.~也就是说,在该波长时,若溶液遵守贝尔-郎比定律,则可以选择这一波长来进行定量分析.~

溴酚蓝在有机溶剂中存在着以下的电离平衡
\begin{equation}
  \text{HA}\rightleftharpoons \text{H}^++\text{A}^-
\end{equation}
其平衡常数
\begin{equation}
  K_a=\frac{\left[\text H^+\right]\left[\text A^-\right]}{\left[\text{HA}\right]}
\end{equation}

溶液的颜色是由显色物质HA与$\text{A}^-$引起的,其变色范围pH在3.1~4.6之间,当pH$\leqslant $3.1时,溶液的颜色主要由HA引起的,呈黄色;在pH$\geqslant $4.6时,溶液的颜色主要由A$^-$引起,呈蓝色.~
用对A$^-$产生最大吸收波长的单色光测定电离后的混合溶液的消光,可求出A$^-$的浓度.~令A$^-$在显色物质中所占的分数为$X$,则HA所占的摩尔分数为$1-X$,所以
\begin{equation}
  K_a=\frac{X}{1-X}\left[\text{A}^-\right]
\end{equation}
即
\begin{equation}\label{1.5}
  \lg\frac{X}{1-X}=\text{pH}+\lg K_a
\end{equation}
根据上式可知,只要测定溶液的pH值及溶液中的$\left[\text{HA}\right]$和$\left[\text{A}^-\right]$,就可以计算出电离平衡常数$K_a$.~

在极酸条件下,HA未电离,此时体系的颜色完全由HA引起,溶液呈黄色.~设此时体系的消光度为$D_1$;在极碱条件下,HA完全电离,此时体系的颜色完全由A$^-$引起,此时的消光度为$D_2$,$D$为两种极端条件之间的诸溶液的消光度,它随着溶液的pH而变化,即
\begin{equation}
  D=\left(1-X\right)D_1+XD_2
\end{equation}
也即
\begin{equation}
  X=\frac{D-D_1}{D_2-D_1}
\end{equation}
代入式(\ref{1.5})中得
\begin{equation}
  \lg\frac{D-D_1}{D_2-D}=\text{pH}-\text pK_a
\end{equation}

在测定$D_1$、$D_2$后,再测一系列pH下的溶液的光密度,以$\lg\frac{D-D_1}{D_2-D}$对pH作图应为一直线,由其在横轴上的截距可求出p$K_a$,从而可得该物质的电离平衡常数.~

\section{实验目的}
\begin{enumerate}
  \item 掌握一种测定酸碱指示剂电离平衡常数的方法;
  \item 熟悉并掌握722型分光光度计的性能和使用方法,并利用分光光度计测量BPB的最大吸收波长,了解溶液浓度对max的影响;
  \item 解酸度对BPB的影响,学会用缓冲溶液调节溶液酸度的方法.~
\end{enumerate}
