% !TeX root = ../main.tex

\chapter{序言}

\section{实验原理}
表面张力$\sigma$满足关系式
\begin{equation}
  -W=\sigma \increment A
\end{equation}
其中$\increment A$为一体系表面增加的面积,$W$为该体系表面增加$\increment A$所需耗费的功.~
因此表面张力$\sigma$单位为$\text{N}\cdot \text{m}^{-1}$.

对于溶液体系,根据能量最低原理,会在溶液表面出现吸附现象,满足关系式
\begin{equation}
  \Gamma = -\frac{c}{RT}\left(\frac{\partial \sigma}{\partial c}\right)_T
\end{equation}
其中$\Gamma~\left(\text{mol}\cdot\text{m}^{-2}\right)~$为气-液界面上的吸附量,
$\sigma~\left(\text{N}\cdot \text{m}^{-1}\right)~$为溶液的表面张力,
$T~\left(\text{K}\right)~$为开尔文温度,$c~\left(\text{mol}\cdot\text{m}^{-3}\right)~$为溶液浓度
$R~\left(=8.314~\text{J}\cdot\text{mol}^{-1}\cdot\text{K}^{-1}\right)~$为气体常数.~

正丁醇为一种表面活性物质$\left(\Gamma>0\right)$,满足其特定的$\sigma-c$曲线.~可根据该曲线切线斜率求出吸附量$\Gamma$,
再根据Langmuir等温方程式
\begin{equation}
  \Gamma=\Gamma_{\infty}\frac{KC}{1+KC}
\end{equation}
可作出$\frac{C}{\Gamma}\sim C$直线
\begin{equation}
  \frac{C}{\Gamma}=\frac{C}{\Gamma_{\infty}}+\frac{1}{K\Gamma_{\infty}}
\end{equation}
由直线斜率即可求出$\Gamma_{\infty}$.~

在饱和吸附时,可求出正丁醇分子横截面积
\begin{equation}
  S_0=\frac{1}{\Gamma_{\infty}\widetilde{N}}
\end{equation}
其中$\widetilde{N}$为阿伏伽德罗常数.

可以通过最大气泡压力法测量表面张力.~最大的压力差$P_{\max}$可在数字式微压差测量仪上读出,满足
\begin{equation}
  P_{\max}=\increment h\rho g
\end{equation}
其中$\increment h$为微压差测量仪读数,$g$为重力加速度,$\rho$为液体的密度.~
因此对两种溶液而言,有
\begin{equation}
  \sigma_1=\frac{\sigma_2}{\increment h_2}\increment h_1=K'\increment h_1.
\end{equation}

\section{实验目的}
通过测定不同浓度正丁醇水溶液的表面张力,由曲线求溶液界面上的吸附量和单个正丁醇分子的横截面积.