% !TeX root = ../main.tex

\ustcsetup{
  keywords = {
    物理化学实验, 表面张力
  },
  keywords* = {
    Physical chemistry experiment, Surface tension
  },
}

\begin{abstract}
  物体表面的分子和内部分子所处的境况不同,因而能量也不同,表面分子的能量比内部分子大.~
  表面张力。它表示表面自动缩小的趋势的大小.~
  表面张力是液体的重要特性之一,与所处的温度、压力、液体的组成共存的另一相的组成等有关.~
  纯液体的表面张力通常指该液体与饱和了其自身蒸气的空气共存的情况而言.~
  根据能量最低原理,溶质能降低溶液的表面张力时,表面层中溶质的浓度应比溶液内部大,反之,溶质使溶液的表面张力升高时,它在表面层中的浓度比在内部的浓度低.~
  这种表面浓度与溶液里面浓度不同的现象叫“吸附”.~
  表面活性物质具有显著的不对称结构,它是由亲水的极性部分和憎水的非极性部分构成.~
  正丁醇就是这样的分子,在水溶液表面的表面活性物质分子,其极性部分朝向溶液内部,而非极性部分朝向空气.~
  当表面张力仪中的毛细管截面与欲测液面相齐时,液面沿毛细管上升.~
  打开滴液漏斗的活塞,使水缓慢下滴而使体系内的压力增加,这时毛细管内的液面上受到一个比恒温试管中液面上稍大的压力,因此毛细管内的液面缓缓下降.~
  当此压力差在毛细管端面上产生的作用力稍大于毛细管口溶液的表面张力时,气泡就从毛细管口逸出.~
  这个最大的压力差可由数字式微压差测量仪上读出.~
  通过测定不同浓度正丁醇水溶液的表面张力,由曲线求溶液界面上的吸附量和单个正丁醇分子的横截面积.~
  了解表面张力的性质、表面能的意义以及表面张力和吸附的关系.~
  掌握一种测定表面张力的方法—最大气泡法.~
\end{abstract}

\begin{abstract*}
  Molecules on the surface of an object are in a different situation, and therefore have a different energy, than molecules in the interior, with surface molecules having more energy than interior molecules.
  Surface Tension. It indicates the magnitude of the tendency of a surface to shrink automatically.
  Surface tension is one of the most important properties of liquids and is related to the temperature, pressure, and composition of the other phase with which the liquid is coexisting.
  The surface tension of a pure liquid usually refers to the coexistence of the liquid with air saturated with its own vapor.
  According to the principle of energy minimization, when a solute lowers the surface tension of a solution, the concentration of the solute in the surface layer should be greater than in the interior of the solution, and conversely, when a solute raises the surface tension of a solution, its concentration in the surface layer is lower than in the interior.
  This difference between the concentration at the surface and the concentration inside the solution is called "adsorption".Adsorption
  Surface-active substances have a remarkable asymmetric structure consisting of a hydrophilic polar portion and a hydrophobic nonpolar portion.
  N-butanol is one such molecule, and the molecules of the surface-active substance on the surface of an aqueous solution have their polar portions oriented towards the interior of the solution and their non-polar portions oriented towards the air.
  When the cross section of the capillary in a surface tension meter is aligned with the surface of the liquid to be measured, the liquid surface rises along the capillary.
  The pressure in the system is increased by opening the piston of the dropping funnel and allowing the water to slowly drip down, the liquid level in the capillary tube is then subjected to a pressure slightly greater than that on the liquid level in the thermostat tube, and so the liquid level in the capillary tube slowly falls.
  When this pressure difference produces a force on the end face of the capillary that is slightly greater than the surface tension of the solution at the mouth of the capillary, bubbles escape from the capillary.
  This maximum pressure difference can be read on a digital differential pressure gauge.
  By measuring the surface tension of aqueous n-butanol solutions of different concentrations, the amount adsorbed at the solution interface and the cross-sectional area of a single n-butanol molecule are determined from the curves.
  Understand the nature of surface tension, the significance of surface energy, and the relationship between surface tension and adsorption.  Understand the significance of surface energy and the relationship between surface tension and adsorption.
  Knowledge of a method for determining surface tension, the maximum bubble method.  Understand the meaning of surface energy and the relationship between surface tension and adsorption.
\end{abstract*}
