%实验数据处理-质量和密度的测量

%文档类型
\documentclass[a4paper]{article}%A4纸,文档类型为论文

%宏包
%文字设置
\usepackage[UTF8]{ctex}%处理中文
\usepackage{xeCJK}%处理中文
\usepackage{fontspec,xunicode,xltxtra}%字体设置
%文档&排版设置
\usepackage{titlesec}%自定义章节标题样式
\usepackage{multicol}%分栏
\usepackage{hyperref}%超链接设置\daleth 
\usepackage{multirow,makecell}%制作复杂表格
\usepackage{booktabs}%画三线表要用
\usepackage{float}%图片、表格等位置浮动排版
\usepackage{indentfirst}%首行缩进
\usepackage{graphicx,subfigure}%图片插入
\usepackage{listings}%代码高亮
\usepackage{xcolor}
\usepackage{appendix}%附录
\usepackage{fancyhdr}%页眉页脚
\usepackage{geometry}%页边距
\usepackage{caption}
%数学
\usepackage{amsmath,amssymb}%公式
\usepackage{amsfonts,mathrsfs,txfonts}%数学字体
\usepackage{array}%矩阵
\usepackage{gensymb}%角度单位“度”的命令:\degree

\geometry{a4paper,scale=0.8}%设置了纸张为a4,并且版心占页面长度的比例为80%;scale也可以改为ratio,表示版面边距占页面长度的比例.
\hypersetup{colorlinks=true,linkcolor=black,citecolor=black}%去掉目录等超链接所带的红框,并让参考文献的引用颜色为黑色
\setlength{\parindent}{2em}%首行缩进2字符
\lstset{
    columns=fixed,
    numbers=left, % 在左侧显示行号
    numberstyle=\footnotesize\color{darkgray},% 设定行号格式
    backgroundcolor=\color[RGB]{245,245,244},% 设定背景颜色
    keywordstyle=\color[RGB]{40,40,255},% 设定关键字颜色
    numberstyle=\color[RGB]{0,192,192},%行号数字样式
    commentstyle=\it\color[RGB]{0,96,96},% 设置代码注释的格式
    stringstyle=\rmfamily\slshape\color[RGB]{128,0,0},% 设置字符串格式
}
\pagestyle{fancy}%页眉页脚
\fancyhead{}%清空页眉
\fancyhead[C]{\emph{实验数据处理:质量和密度的测量}}


\newcommand{\kaiti}{\CJKfamily{STKaiti}} % 楷体:\kaiti或\emph
\newcommand{\fs}{\CJKfamily{STFangsong}} % 仿宋:\fs或\fangsong
%黑体:\heiti

\newcommand{\suo}{\indent}%缩进2个字符

\title{\heiti{实验报告}}%标题
\author{{\emph{李佩哲}}}
\date{\emph{\small\today}}

\begin{document}
\begin{center}
\center{\bf{\LARGE{实验数据处理}\\\large{质量和密度的测量}}}\\
\emph{李佩哲~~~PB21051049~~~\\\today}
\end{center}
\section{测量记录}
原始数据见附件.\\整理如下

\emph{称量金属圆柱的尺寸和质量:}$D=2.480~\text{cm},~H=3.990~\text{cm},~m=163.64~\text{g}$

\emph{排水法测金属圆柱的体积:}$\frac{F_\text{浮}}{g}=19.23~\text{g}$

\emph{弹簧振子法测金属片质量:}见表\ref{1}

\emph{转动定律法测小圆柱质量:}小金属块$2m=29.92~$g,其余见表\ref{2}
\begin{table}[H]
    \begin{minipage}{0.5\linewidth}
        \centering
        \begin{tabular}{ccc}
            \toprule
            $m$/g & $t(30T)$/s\\
            \midrule
            $m_0$&37.21\\
            $m_0+100.01$&52.75\\
            $m_0+m_x$&45.74\\
            \bottomrule
        \end{tabular}
        \caption{弹簧振子法测金属片质量}\label{1}
    \end{minipage}
    \begin{minipage}{0.5\linewidth}  
        \centering
        \begin{tabular}{ccc} 
            \toprule
            $r$/cm & $t(30T)$/s\\
            \midrule
            10.00&78.72\\
            20.00&61.14\\
            30.00&56.00\\
            40.00&54.63\\
            50.00&55.37\\
            \midrule
            $r_x=$40.00&$t_x=$48.74\\
            \bottomrule
        \end{tabular}
        \caption{转动定律法测小圆柱质量}\label{2}
    \end{minipage}
\end{table}

\section{分析与讨论}
\subsection{金属圆柱}
质量$m=0.16364~$kg,体积由$\rho gV=F_\text{浮}$得$V=1.9287\times 10^{-5}~\text{m}^3$。故密度$\rho =\frac{m}{V}=8.4845\times 10^{3}~$kg/m$^3$.

另外卡尺法所测体积$V^\prime=\pi\left(\frac{D}{2}\right)^2H=1.9274\times 10^{-5}~\text{m}^3$。故此法所得密度$\rho^\prime=\frac{m}{V^\prime}=8.4903\times 10^{3}~$kg/m$^3$.
\subsection{金属片}
由$m_0+100.01=\left(\frac{52.75}{37.21}\right)^2m_0$得$m_0=99.05~$g,从而由$m_x+m_0=\left(\frac{45.74}{37.21}\right)^2m_0$得$m_x=50.92~$g.

\subsection{小圆柱}
由$\frac{gr}{4\pi^2}T^2=r^2+\frac{I_c}{2m}$得$I_c=0.005014$,故$m=\frac{I_c}{\frac{T^2}{4\pi^2}gR-\frac{1}{12}L^2-R^2}=49.59~$g.

\end{document}