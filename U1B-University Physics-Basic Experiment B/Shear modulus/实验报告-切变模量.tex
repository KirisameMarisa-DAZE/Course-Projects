%实验报告-切变模量

%文档类型
\documentclass[a4paper]{article}%A4纸,文档类型为论文

%宏包
%文字设置
\usepackage[UTF8]{ctex}%处理中文
\usepackage{xeCJK}%处理中文
\usepackage{fontspec,xunicode,xltxtra}%字体设置
%文档&排版设置
\usepackage{titlesec}%自定义章节标题样式
\usepackage{multicol}%分栏
\usepackage{hyperref}%超链接设置\daleth 
\usepackage{multirow,makecell}%制作复杂表格
\usepackage{booktabs}%画三线表要用
\usepackage{float}%图片、表格等位置浮动排版
\usepackage{indentfirst}%首行缩进
\usepackage{graphicx,subfigure}%图片插入
\usepackage{listings}%代码高亮
\usepackage{xcolor}
\usepackage{appendix}%附录
\usepackage{fancyhdr}%页眉页脚
\usepackage{geometry}%页边距
\usepackage{caption}
%数学
\usepackage{amsmath,amssymb}%公式
\usepackage{amsfonts,mathrsfs,txfonts}%数学字体
\usepackage{array}%矩阵
\usepackage{gensymb}%角度单位“度”的命令:\degree

\geometry{a4paper,scale=0.8}%设置了纸张为a4,并且版心占页面长度的比例为80%;scale也可以改为ratio,表示版面边距占页面长度的比例.
\hypersetup{colorlinks=true,linkcolor=black,citecolor=black}%去掉目录等超链接所带的红框,并让参考文献的引用颜色为黑色
\setlength{\parindent}{2em}%首行缩进2字符
\lstset{
    columns=fixed,
    numbers=left, % 在左侧显示行号
    numberstyle=\footnotesize\color{darkgray},% 设定行号格式
    backgroundcolor=\color[RGB]{245,245,244},% 设定背景颜色
    keywordstyle=\color[RGB]{40,40,255},% 设定关键字颜色
    numberstyle=\color[RGB]{0,192,192},%行号数字样式
    commentstyle=\it\color[RGB]{0,96,96},% 设置代码注释的格式
    stringstyle=\rmfamily\slshape\color[RGB]{128,0,0},% 设置字符串格式
}
\pagestyle{fancy}%页眉页脚
\fancyhead{}%清空页眉
\fancyhead[C]{\emph{实验报告II:切变模量}}


\newcommand{\kaiti}{\CJKfamily{STKaiti}} % 楷体:\kaiti或\emph
\newcommand{\fs}{\CJKfamily{STFangsong}} % 仿宋:\fs或\fangsong
%黑体:\heiti

\newcommand{\suo}{\indent}%缩进2个字符

\title{\heiti{实验报告}}%标题
\author{{\emph{李佩哲}}}
\date{\emph{\small\today}}

\begin{document}
\begin{center}
\center{\bf{\LARGE{实验报告}\\\large{切变模量}}}\\
\emph{李佩哲~~~PB21051049~~~\\\today}
\end{center}
\section{实验目的}
测量金属丝的扭转模量与切变模量.

\section{原理}
根据剪切胡克定律$$\tau=G\gamma$$有$$\tau_\rho=G\gamma_\rho=G\rho \frac{\text d\phi}{\text dl}$$
于是横截面上距轴线距离$\rho$处切应力恢复力矩为$$\tau \rho 2\pi\rho\text d\rho=2\pi G\rho^3\frac{\text d\phi}{\text dl}\text d\rho$$
从而总恢复力矩为$$M=\frac{\pi}{2}GR^4\frac{\phi}{L}$$
故$$G=\frac{2ML}{\pi R^4\phi}$$
让金属丝与摆进行转动,根据简谐运动的规律,有$$T_0=2\pi \sqrt{\frac{I_0}{D}}$$
其中$I_0$为摆的转动惯量.~为了便于测量与计算,在圆盘上放置一质量为$m$的金属环,则扭摆的周期变为$$T_1=2\pi\sqrt{\frac{I_0+I_1}{D}}$$
结合之前的式子,最终可得$$D=\frac{2\pi^2m\left(r_{\text{内}}^2+r_{\text{外}}^2\right)}{T_1^2-T_0^2}$$
$$G=\frac{4\pi Lm\left(r_{\text{内}}^2+r_{\text{外}}^2\right)}{R^4\left(T_1^2-T_0^2\right)}$$
根据这个式子,扭转模量$D$与切变模量$G$可求.

\section{实验仪器}
扭摆、金属环、秒表等.

\section{测量记录}
原始数据见附件.\\整理如下
\begin{table}[H]
    \begin{minipage}{0.2\linewidth}
        \centering
        \begin{tabular}{ccc}
            \toprule
            数据  & 值\\
            \midrule
            $L$/m & 0.4420\\
            $m$/kg & 0.5645\\
            $d_\text{内}$/m & 0.08402\\
            $d_\text{外}$/m & 0.10378\\
            \bottomrule
        \end{tabular}
        \caption{次要误差数据}\label{1}
    \end{minipage}
    \begin{minipage}{0.4\linewidth}  
        \centering
        \begin{tabular}{ccc} 
            \toprule
            序号 &零误差/mm & $d$/m\\
            \midrule
            1&0.000&0.000781\\
            2&0.000&0.000779\\
            3&0.000&0.000779\\
            4&0.000&0.000776\\
            5&0.000&0.000776\\
            6&0.000&0.000776\\
            7&0.000&0.000775\\
            8&0.000&0.000778\\
            9&0.000&0.000779\\
            \bottomrule
        \end{tabular}
        \caption{金属丝直径}\label{2}
    \end{minipage}
    \begin{minipage}{0.3\linewidth}  
        \centering
        \begin{tabular}{ccc} 
            \toprule
            序号 &$t_0\left(\pi,50T\right)$/s & $t_1\left(\pi,80T\right)$/s\\
            \midrule
            1&301.29&112.66\\
            2&301.25&112.73\\
            3&301.32&112.81\\
            4&301.36&112.79\\
            5&301.32&112.80\\
            6&301.31&112.89\\
            \bottomrule
        \end{tabular}
        \caption{总时间}\label{3}
    \end{minipage}
\end{table}

\section{分析与讨论}
\subsubsection{数据处理}
由表\ref{1}可知,$L=0.4420~\text m$,$m=0.5645~\text{kg}$,$r_{\text 内}=0.04201~\text m$,$r_{\text 外}=0.05189~\text m$.

由表\ref{2}可知,金属丝半径$R\approx0.00038883~\text m$.

由表\ref{3}可知,$T_1\approx3.7664~\text s$,$T_0\approx2.2556~\text s$.

所以$D=\frac{2\pi^2m\left(r_{\text{内}}^2+r_{\text{外}}^2\right)}{T_1^2-T_0^2}=5.4594\times 10^{-3}~$N$\cdot$m,
$G=\frac{4\pi Lm\left(r_{\text{内}}^2+r_{\text{外}}^2\right)}{R^4\left(T_1^2-T_0^2\right)}\approx67.204$~GPa.


\subsubsection{误差分析}
由$D=\frac{2\pi^2m\left(r_{\text{内}}^2+r_{\text{外}}^2\right)}{T_1^2-T_0^2}$知,相对不确定度
\begin{equation*}
    \begin{aligned}
        \frac{\Delta D}{D}&=\sqrt{
            \left(\frac{\Delta m}{m}\right)^2
            +\left(\frac{2r_{\text{内}}\Delta r_{\text{内}}}{r_{\text{内}}^2+r_{\text{外}}^2}\right)^2
            +\left(\frac{2r_{\text{外}}\Delta r_{\text{外}}}{r_{\text{内}}^2+r_{\text{外}}^2}\right)^2
            +\left(\frac{2T_1\Delta t}{n_1\left(T_1^2-T_0^2\right)}\right)^2
            +\left(\frac{2T_0\Delta t}{n_0\left(T_1^2-T_0^2\right)}\right)^2}\\
        &=\sqrt{
            \left(\frac{0.1461}{564.5}\right)^2
            +\left(\frac{2\times 0.04201\times 2.124\times 10^{-5}}{0.04201^2+0.05189^2}\right)^2
            +\left(\frac{2\times 0.05189\times 2.124\times 10^{-5}}{0.04201^2+0.05189^2}\right)^2}\\
            &\overline{
            +\left[\frac{2\times 3.7664\times 0.074968}{80\left(3.7664^2-2.2556^2\right)}\right]^2
            +\left[\frac{2\times 2.2556\times 0.035621}{50\left(3.7664^2-2.2556^2\right)}\right]^2
            }=0.109472\%
    \end{aligned}
\end{equation*}
由$G=\frac{4\pi Lm\left(r_{\text{内}}^2+r_{\text{外}}^2\right)}{R^4\left(T_1^2-T_0^2\right)}$
知,相对不确定度
\begin{equation*}
    \begin{aligned}
        \frac{\Delta G}{G}&=\sqrt{
            \left(\frac{\Delta L}{L}\right)^2
            +\left(\frac{\Delta m}{m}\right)^2
            +\left(\frac{2r_{\text{内}}\Delta r_{\text{内}}}{r_{\text{内}}^2+r_{\text{外}}^2}\right)^2
            +\left(\frac{2r_{\text{外}}\Delta r_{\text{外}}}{r_{\text{内}}^2+r_{\text{外}}^2}\right)^2
            +\left(\frac{4\Delta R}{R}\right)^2
            +\left(\frac{2T_1\Delta t}{n_1\left(T_1^2-T_0^2\right)}\right)^2
            +\left(\frac{2T_0\Delta t}{n_0\left(T_1^2-T_0^2\right)}\right)^2}\\
        &=\sqrt{
            \left(\frac{0.001021}{0.4420}\right)^2
            +\left(\frac{0.1461}{564.5}\right)^2
            +\left(\frac{2\times 0.04201\times 2.124\times 10^{-5}}{0.04201^2+0.05189^2}\right)^2
            +\left(\frac{2\times 0.05189\times 2.124\times 10^{-5}}{0.04201^2+0.05189^2}\right)^2}\\
            &\overline{
            +\left(\frac{4\times 4.246\times 10^{-6}}{0.0003883}\right)^2
            +\left[\frac{2\times 3.7664\times 0.074968}{80\left(3.7664^2-2.2556^2\right)}\right]^2
            +\left[\frac{2\times 2.2556\times 0.035621}{50\left(3.7664^2-2.2556^2\right)}\right]^2
            }=4.38129\%
    \end{aligned}
\end{equation*}
故$\Delta D=0.109472\%D=5.977\times 10^{-6}~$N$\cdot$m,$\Delta G=4.38129\%G=2.944~$GPa

综上,$D=5.4594\pm 0.005977~$N$\cdot$mm,$G=67.204\pm 2.944~$GPa

\section{思考}
位了提高实验精度,本实验对主要误差的来源的物理量进行了多次测量,并在设计实验时结合已知数据进行适当的估算。

在具体测量时应注意控制变量,保持金属丝伸直、圆盘转动水平等等。·

\end{document}