% !TeX root = ../main.tex

\chapter{绪论}
公元1858年,两名德国数学家莫比乌斯和Johann Benedict Listing分别发现,一个扭转180度后再两头粘接起来的纸条,具有魔术般的性质。
与普通纸带具有两个面(双侧曲面)不同,这样的纸带只有一个面(单侧曲面),一只小虫可以爬遍整个曲面而不必跨过它的边缘!
这一神奇的单面纸带被称为“莫比乌斯带”。作为一种典型的拓扑图形,莫比乌斯带引起了许多科学家的研究兴趣,并在生活和生产中有了一些应用。
例如,动力机械的皮带就可以做成“莫比乌斯带”状,这样皮带就不会只磨损一面了。
\cite{ref0}
实验室中也有可能产生莫比乌斯带形状的粒子。
一群科学家在Journal of Chemical Physics上发表了一篇论文,其中预言了一种莫比乌斯带形状的碳单质(准确来说应该是石墨烯)。
它能抵抗摄氏200度左右的温度,算是相当稳定。由于它莫比乌斯带的结构,它应该是一个偶极子,从而可以形成稳定的晶体。
\cite{ref00}

鉴于石墨烯可以带电,现假设某由石墨烯构成的莫比乌斯带均匀带电,以此来研究其周围空间中电场强度和电势的分布情况。
对于一个带电的物体,在计算它的电场时,可以把它分成若干小块,只要每个小块足够小,就可把每小块所带的电荷看成为点电荷,然后用点电荷电场叠加的方法计算整个带电体的电场。
\cite{ref01}