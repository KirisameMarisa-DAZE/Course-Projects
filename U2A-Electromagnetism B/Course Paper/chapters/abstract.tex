% !TeX root = ../main.tex

\ustcsetup{
  keywords = {
    电磁学, 莫比乌斯带, 电场分布
  },
  keywords* = {
    Electromagnetics, Mobius strip, Electric field distribution
  },
}

\begin{abstract}
  莫比乌斯带的性质十分奇特。为了学以致用,探究莫比乌斯带周围电场的分布情况,本实验通过Matlab模拟的方法对莫比乌斯带周围电场强度、电势高低进行了计算,并进行了可视化。

  通过观察结果图像,课题组发现,靠近莫比乌斯带面的地方,电场强度矢量基本垂直于该面切平面,与法向量平行。此外,$y$坐标大的地方,电场强度更大;距中心圆点相同距离的点$x$坐标大的,电势更高。

  本实验通过一种通用的求解积分的方法对莫比乌斯带的性质进行了探究,因此此法也可以进行改良,制作一种仅需输入研究对象参数方程即可求解出该对象的电场电势分布状况的一般性程序。
  万分遗憾的是,碍于时间与资源的限制,课题组未能对此法做出卓有成效的尝试,希望其他研究者可以再接再厉,勇创新高。
\end{abstract}

\begin{abstract*}
  The nature of the Mobius belt is very peculiar. In order to apply what we have learned and explore the distribution of the electric field around the Mobius Belt, 
  this experiment calculated the electric field strength and potential around the Mobius Belt through the method of Matlab simulation, and visualized them. 
  
  Through the observation of the result image, 
  the research group found that the electric field intensity vector is basically perpendicular to the tangent plane of the Mobius belt surface and parallel to the normal vector. 
  In addition, where the $y $coordinate is large, the electric field intensity is greater; The point with the same distance from the center dot $x $coordinates is larger, and the potential is higher. 
  
  In this experiment, we explored the properties of Mobius band through a general method of solving integral, so this method can also be improved, 
  and a general program can be made to solve the distribution of electric field potential of the object by inputting the parameter equation of the object of study. 
  It is a great pity that due to the limitation of time and resources, the research group failed to make an effective attempt on this method. 
  We hope that other researchers can continue to work hard to reach new heights.
\end{abstract*}
