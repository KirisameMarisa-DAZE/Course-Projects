% !TeX root = ../main.tex

\chapter{构建方法}

\section{莫比乌斯带参数方程及模型的构建}
\subsection{莫比乌斯带的参数方程\cite{ref1}}
认为莫比乌斯带是通过一条长为$2t_0$线段$AB$绕自身中点$M$旋转,同时中点$M$以$R$为半径绕坐标原点$O$旋转而成。那么在此过程中,线段$AB$绕自身中点$M$旋转了$\pi$,
同时中点$M$绕坐标原点$O$旋转了$2\pi$。通常认为旋转是均匀进行的,那么线段$AB$旋转的转角$\theta$与中点$M$转过的极角$\phi$满足如下关系
\begin{equation}
  \theta=\frac{1}{2}\phi
\end{equation}
那么线段$AB$中点$M$的坐标可用参数方程表示为
\begin{equation}
  \begin{cases}
    x=R\cos \phi\\
    y=R\sin\phi\\
    z=0
  \end{cases}
\end{equation}

另设一参量$t$表示线段上某点$P\left(x_0,y_0,z_0\right)$到线段$AB$中点$M$的距离,那么,$P$点的坐标可以通过$M$点的坐标写出为
\begin{equation}
  \begin{cases}
    x_0=R\cos \phi+t\cos\theta\cos\phi=\left(R+t\cos\frac{\phi}{2}\right)\cos\phi\\
    y_0=R\sin\phi+t\cos\theta\sin\phi=\left(R+t\cos\frac{\phi}{2}\right)\sin\phi\\
    z_0=0+t\sin\theta=t\sin\frac{\phi}{2}
  \end{cases}
\end{equation}
其中$t\in\left[-t_0,t_0\right]$,$\phi\in\left[0,2\pi\right)$。由此可以得出莫比乌斯带的参数方程
\begin{equation}\label{canshufangcheng}
  \begin{cases}
    x=\left(R+t\cos\frac{\phi}{2}\right)\cos\phi\\
    y=\left(R+t\cos\frac{\phi}{2}\right)\sin\phi\\
    z=t\sin\frac{\phi}{2}
  \end{cases}
\end{equation}

\subsection{莫比乌斯带模型的生成}
根据方程(\ref{canshufangcheng}),可以通过如下Matlab程序绘制莫比乌斯带的图像
\begin{lstlisting}[language={Matlab},
  numbers=left,
  numberstyle=\tiny\menlo,
  basicstyle=\small\menlo][H]
clear;
clc;
t=linspace(-0.5,0.5,40);
phi=linspace(0,2.*pi,40);
[T,Phi]=meshgrid(t,phi);
x=(1+T.*cos(Phi./2)).*cos(Phi);
y=(1+T.*cos(Phi./2)).*sin(Phi);
z=T.*sin(Phi./2);
surf(x,y,z);
colormap spring;
\end{lstlisting}


\section{电场强度及电势的计算\cite{ref2}}
\subsection{电场强度的计算}
\begin{equation}
  E=\left(\frac{\partial x}{\partial t}\right)^2+\left(\frac{\partial y}{\partial t}\right)^2+\left(\frac{\partial z}{\partial t}\right)^2=1
\end{equation}
\begin{equation}
  G=\left(\frac{\partial x}{\partial \phi}\right)^2+\left(\frac{\partial y}{\partial \phi}\right)^2+\left(\frac{\partial z}{\partial \phi}\right)^2
\end{equation}
\begin{equation}
  F=\frac{\partial x}{\partial t}\frac{\partial x}{\partial \phi}+\frac{\partial y}{\partial t}\frac{\partial y}{\partial \phi}+\frac{\partial z}{\partial t}\frac{\partial z}{\partial \phi}
\end{equation}
所以场强
\begin{equation}
  E=\frac{\sigma}{4\pi\epsilon_0}\iint_D\frac{\sqrt{EG-F^2}}{\sqrt{x_0^2+y_0^2+z_0^2}}\dif t\dif \phi
\end{equation}
其中$D=\left\{-\frac{1}{2}\le t\le \frac{1}{2},~0\le \phi\le\ 2\pi\right\}$。
\begin{lstlisting}[language={Matlab},
  numbers=left,
  numberstyle=\tiny\menlo,
  basicstyle=\small\menlo][H]
function [Ex,Ey,Ez,E] = Ee(R,x0,y0,z0)
% 输入莫比乌斯带半径R及空间中某点坐标(x0,y0,z0)
% 输出该点场强E及其分量Ex,Ey,Ez

% 给出莫比乌斯带的参数方程
syms t phi;
x=(1+t.*cos(phi./2)).*cos(phi);
y=(1+t.*cos(phi./2)).*sin(phi);
z=t.*sin(phi./2);
% 给出空间中一点到面上某点的距离r
r=sqrt((x-x0).^2+(y-y0).^2+(z-z0).^2);
% 面元dS的面积元素(E==1)
G=(-sin(phi)-0.5.*t.*sin(phi./2).*cos(phi)-t.*cos(phi./2).*sin(phi)).^2+(cos(phi)-0.5.*t.*sin(phi./2)+t.*cos(phi./2).*cos(phi)).^2+(0.5.*t.*cos(phi./2)).^2;
F=-0.5.*t.*sin(phi).*(cos(phi).^2);

% 各个分量的需要积分的部分
funx=sqrt(G-F.^2).*(x0-x)./(r.^3);
fx=matlabFunction(funx);% 参数化
funy=sqrt(G-F.^2).*(y0-y)./(r.^3);
fy=matlabFunction(funy);
funz=sqrt(G-F.^2).*(z0-z)./(r.^3);
fz=matlabFunction(funz);

% 各个分量需要积分的部分的积分
Funx=integral2(fx,-0.5,0.5,0,2.*pi);
Funy=integral2(fy,-0.5,0.5,0,2.*pi);
Funz=integral2(fz,-0.5,0.5,0,2.*pi);
% 总场强需要积分的部分的积分
Fun=sqrt(Funx.^2+Funy.^2+Funz.^2);

% σ, ε_0
sigma=1;
epsilon_0=8.854187817e-12;

% 场强及其分量
Ex=sigma./(4.*pi.*epsilon_0).*Funx;
Ey=sigma./(4.*pi.*epsilon_0).*Funy;
Ez=sigma./(4.*pi.*epsilon_0).*Funz;
E=sigma./(4.*pi.*epsilon_0).*Fun;
end  
\end{lstlisting}

\subsection{电势的计算}
同理,电势
\begin{equation}
  U=\frac{\sigma}{4\pi\epsilon_0}\iint_D\frac{\sqrt{EG-F^2}}{\left[\left(x-x_0\right)^2+\left(y-y_0\right)^2+\left(z-z_0\right)^2\right]^\frac{3}{2}}\dif t\dif \phi
\end{equation}
其中$D=\left\{-\frac{1}{2}\le t\le \frac{1}{2},~0\le \phi\le\ 2\pi\right\}$。
\begin{lstlisting}[language={Matlab},
numbers=left,
numberstyle=\tiny\menlo,
basicstyle=\small\menlo][H]
function [U] = Uu(R,x0,y0,z0)
% 输入莫比乌斯带半径R及空间中某点坐标(x0,y0,z0)
% 输出该点电势U

% 给出莫比乌斯带的参数方程
syms t phi;
x=(R+t.*cos(phi./2)).*cos(phi);
y=(R+t.*cos(phi./2)).*sin(phi);
z=t.*sin(phi./2);
% 给出空间中一点到面上某点的距离r
r=sqrt((x-x0).^2+(y-y0).^2+(z-z0).^2);

% 面元dS的面积元素(E==1)
G=(-sin(phi)-0.5.*t.*sin(phi./2).*cos(phi)-t.*cos(phi./2).*sin(phi)).^2+(cos(phi)-0.5.*t.*sin(phi./2)+t.*cos(phi./2).*cos(phi)).^2+(0.5.*t.*cos(phi./2)).^2;
F=-0.5.*t.*sin(phi).*(cos(phi).^2);

% 电势的需要积分的部分
fun=sqrt(G-F.^2)./r;
f=matlabFunction(fun);% 参数化

% 电势需要积分的部分的积分
Fun=integral2(f,-0.5,0.5,0,2.*pi);

% σ, ε_0
sigma=1;
epsilon_0=8.854187817e-12;

% 电势
U=sigma./(4.*pi.*epsilon_0).*Fun;
end  
\end{lstlisting}

\section{电场强度与电势的可视化}
\subsection{电场强度的可视化}
在$\left[-5,5\right]\cap\left[-5,5\right]\cap\left[-5,5\right]$以1为间隔取125个点,分别计算他们的电场强度,并画在图上
\begin{lstlisting}[language={Matlab},
  numbers=left,
  numberstyle=\tiny\menlo,
  basicstyle=\small\menlo][H]
clear;clc;

R=1;i=1;j=1;k=1;
Ex=zeros(5,5,5);
Ey=zeros(5,5,5);
Ez=zeros(5,5,5);
E=zeros(5,5,5);

for x=-2:1:2
    for y=-2:1:2
        for z=-2:1:2
            [ex,ey,ez,e]=Ee(R,x,y,z);
            Ex(i,j,k)=ex;
            Ey(i,j,k)=ey;
            Ez(i,j,k)=ez;
            E(i,j,k)=e;
            if (e>10e10)
                Ex(i,j,k)=0;
                Ey(i,j,k)=0;
                Ez(i,j,k)=0;
                E(i,j,k)=0;
            end
            disp(i)
            disp(j)
            disp(k)
            fprintf('\n');
            k=k+1;
        end
        j=j+1;
        k=1;
    end
    i=i+1;
    j=1;
    k=1;
end


x=[-2:1:2];
y=[-2:1:2];
z=[-2:1:2];
[X,Y,Z] = meshgrid(x,y,z);
hold on
quiver3(X,Y,Z,Ex,Ey,Ez);
grid on;
axis equal;

t=linspace(-0.5,0.5,40);
phi=linspace(0,2.*pi,40);
[T,Phi]=meshgrid(t,phi);
x=(1+T.*cos(Phi./2)).*cos(Phi);
y=(1+T.*cos(Phi./2)).*sin(Phi);
z=T.*sin(Phi./2);
surf(x,y,z);
colormap spring;

grid on;%网格线
xlabel('x');%x轴名称
ylabel('y');%y轴名称
zlabel('z');%z轴名称
hold off
\end{lstlisting}

\subsection{电势的可视化}
在$\left[-5,5\right]\cap\left[-5,5\right]\cap\left[-5,5\right]$以1为间隔取125个点,分别计算他们的电势,并画在图上
\begin{lstlisting}[language={Matlab},
  numbers=left,
  numberstyle=\tiny\menlo,
  basicstyle=\small\menlo][H]
clear;clc;

R=1;i=1;j=1;k=1;

U=zeros(1,125);
for x=-2:1:2
    for y=-2:1:2
        for z=-2:1:2
            [u]=Uu(R,x,y,z);
            U((i-1)*25+(j-1)*5+k)=u;
            disp(i)
            disp(j)
            disp(k)
            fprintf('\n');
            k=k+1;
        end
        j=j+1;
        k=1;
    end
    i=i+1;
    j=1;
    k=1;
end

X=zeros(1,125);
Y=zeros(1,125);
Z=zeros(1,125);
for i=1:1:5
    for j=1:1:5
        for k=1:1:5
            X((i-1)*25+(j-1)*5+k)=i-3;
            Y((i-1)*25+(j-1)*5+k)=j-3;
            Z((i-1)*25+(j-1)*5+k)=k-3;
        end
    end
end


hold on
scatter3(X,Y,Z,U.*(10.^(-8.5)),U,'filled');
axis equal;

t=linspace(-0.5,0.5,40);
phi=linspace(0,2.*pi,40);
[T,Phi]=meshgrid(t,phi);
x=(1+T.*cos(Phi./2)).*cos(Phi);
y=(1+T.*cos(Phi./2)).*sin(Phi);
z=T.*sin(Phi./2);
surf(x,y,z);
colormap spring;


colorbar;%颜色栏图例
grid on;%网格线
xlabel('x');%x轴名称
ylabel('y');%y轴名称
zlabel('z');%z轴名称
hold off
\end{lstlisting}