% !TeX root = ../main.tex

\chapter{实验模型}
\section{模型介绍}
本文基于R语言设计相应的模型,使用read\_csv()函数读取数据文件JPY\_CNY.csv。
在读取数据以后,按以下步骤依次进行模型建构与预测:
\begin{enumerate}
    \item \textbf{数据清洗}:转换日期格式为Date类型,去除缺失值,检查并处理重复值,确保收盘价和涨跌幅为数值型数据。
    \item \textbf{描述性统计分析}:计算收盘价和涨跌幅的均值、标准差、最小值、最大值、中位数。
    \item \textbf{时间序列分析}:使用移动平均平滑数据,观察长期趋势;使用ARIMA模型预测未来汇率变化。
    \item \textbf{波动性分析}:使用GARCH和EGARCH模型分析汇率的波动性。
    \item \textbf{风险分析}:使用VaR方法评估汇率波动带来的潜在损失;进行压力测试。
\end{enumerate}

\section{模型代码}
\begin{lstlisting}
    # 加载必要的包
    library(xts)
    library(zoo)
    library(tseries)
    library(rugarch)
    library(forecast)
    library(tidyverse)
    library(lubridate)
    library(PerformanceAnalytics)
    
    
    # 数据读取与清洗
    
    ### 读取CSV文件
    data <- read_csv("JPY_CNY.csv")
    
    ### 检查并修正列名
    colnames(data) <- c("Date", "Close", "Open", "High", "Low", "Volume", "Change")
    
    ### 数据清洗
    data <- data %>%
      # 转换日期格式
      mutate(Date = ymd(Date)) %>%
      # 去除缺失值
      drop_na(Close) %>%
      # 检查并处理重复值
      distinct() %>%
      # 确保收盘价和涨跌幅为数值型数据
      mutate(
        Close = as.numeric(Close),
        Open = as.numeric(Open),
        High = as.numeric(High),
        Low = as.numeric(Low),
        Change = as.numeric(str_replace(Change, "%", "")) / 100
      )
    
    
    # 描述性统计分析
    summary_stats <- data %>%
      summarise(
        Mean_Close = mean(Close, na.rm = TRUE),
        SD_Close = sd(Close, na.rm = TRUE),
        Min_Close = min(Close, na.rm = TRUE),
        Max_Close = max(Close, na.rm = TRUE),
        Median_Close = median(Close, na.rm = TRUE),
        Mean_Change = mean(Change, na.rm = TRUE),
        SD_Change = sd(Change, na.rm = TRUE),
        Min_Change = min(Change, na.rm = TRUE),
        Max_Change = max(Change, na.rm = TRUE),
        Median_Change = median(Change, na.rm = TRUE)
      )
    
    ### 显示描述性统计结果
    print(summary_stats)
    
    
    # 时间序列分析
    
    ### 创建xts对象
    xts_data <- xts(data$Close, order.by = data$Date)
    
    ### 移动平均
    ##### 计算12个月的移动平均
    ma_12 <- rollmean(xts_data, k = 12, align = "right", fill = NA)
    ##### 绘制时间序列图
    plot(xts_data, main = "收盘价时间序列与移动平均线", col = "blue", major.ticks = "years", minor.ticks = FALSE)
    lines(ma_12, col = "red", lwd = 2)
    legend("topright", legend = c("收盘价", "12个月移动平均"), col = c("blue", "red"), lty = 1)
    
    ### ARIMA模型
    ##### 拟合ARIMA模型
    fit <- auto.arima(xts_data)
    ##### 生成预测
    forecast_data <- forecast(fit, h = 24) # 预测未来24个数据点(例如,2年,如果是按月数据)
    ##### 获取最近15年的数据
    end_date <- end(xts_data)
    start_date <- end_date - years(15)
    xts_data_last_10_years <- window(xts_data, start = start_date, end = end_date)
    ##### 获取对应的预测数据
    forecast_dates <- seq(end_date + months(1), by = "month", length.out = 24)
    forecast_data_xts <- xts(forecast_data$mean, order.by = forecast_dates)
    ##### 合并实际数据和预测数据
    combined_data <- merge(xts_data_last_10_years, forecast_data_xts, all = TRUE)
    combined_data <- na.locf(combined_data, fromLast = TRUE)
    ##### 绘制实际数据的时间序列图
    plot(index(combined_data), coredata(combined_data[, 1]), type = "l", main = "汇率的ARIMA模型预测", col = "black", xlab = "时间", ylab = "汇率")
    ##### 添加预测值和置信区间
    lines(index(forecast_data_xts), coredata(forecast_data_xts), col = "red", lty = 2)
    lines(index(forecast_data_xts), forecast_data$lower[, 2], col = "red", lty = 3)
    lines(index(forecast_data_xts), forecast_data$upper[, 2], col = "red", lty = 3)
    ##### 添加图例
    legend("topleft", legend = c("实际数据", "预测值", "置信区间"), col = c("black", "red", "red"), lty = c(1, 2, 3), lwd = 2)
    
    
    # 波动性分析
    
    ### GARCH模型
    ##### 构建GARCH(1, 1)模型
    spec <- ugarchspec(variance.model = list(model = "sGARCH", garchOrder = c(1, 1)), mean.model = list(armaOrder = c(1, 1)))
    ##### 拟合模型
    fit_garch <- ugarchfit(spec = spec, data = xts_data)
    ##### 模型摘要
    summary(fit_garch)
    ##### 预测未来波动性
    forecast_garch <- ugarchforecast(fit_garch, n.ahead = 12)
    ##### 绘制预测结果
    plot(forecast_garch, which=3)
    
    ### EGARCH模型
    ##### 构建EGARCH(1, 1)模型
    spec_egarch <- ugarchspec(variance.model = list(model = "eGARCH", garchOrder = c(1, 1)), mean.model = list(armaOrder = c(1, 1)))
    ##### 拟合模型
    fit_egarch <- ugarchfit(spec = spec_egarch, data = xts_data)
    ##### 模型摘要
    summary(fit_egarch)
    ##### 预测未来波动性
    forecast_egarch <- ugarchforecast(fit_egarch, n.ahead = 12)
    ##### 绘制预测结果
    plot(forecast_egarch, which=3)
    
    
    # 风险分析
    
    ### VaR分析
    ##### 计算每日收益率
    returns <- diff(log(xts_data))
    ##### 计算VaR
    VaR_95 <- VaR(returns, p = 0.95, method = "historical")
    ##### 显示VaR结果
    print(VaR_95)
    
    ### 情景分析和压力测试
    ##### 定义一个极端情景
    extreme_scenario <- returns - 0.05
    ##### 压力测试
    stress_test <- VaR(extreme_scenario, p = 0.95, method = "historical")
    ##### 显示压力测试结果
    print(stress_test)
\end{lstlisting}
