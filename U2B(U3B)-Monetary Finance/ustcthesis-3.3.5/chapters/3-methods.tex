% !TeX root = ../main.tex

\chapter{研究方法}

\section{数据收集}
首先,本文从公开的金融数据源(如彭博、雅虎财经等)收集2009年5月至2024年5月期间的人民币兑日元汇率日数据。数据包括每日的开盘价、收盘价、最高价和最低价。为了保证数据的准确性和完整性,本文对不同来源的数据进行了交叉验证。

\section{数据预处理}
\subsection{缺失值处理}
针对数据中的缺失值,本文采用线性插值法进行填补,保证时间序列的连续性。同时对异常值进行了去除。
\subsection{数据平滑}
为了消除数据中的噪声,本文对汇率数据进行了移动平均处理,选取合适的窗口期,以平滑数据并突出长期趋势。
\subsection{数据标准化}
对数据进行标准化处理,使其符合统计分析的要求,便于模型的构建和分析。清洗后的数据被创建为时间序列对象,以便进行后续的时间序列分析和波动性分析。

\section{时间序列分析}
\subsection{移动平均}
为了观察人民币兑日元汇率的长期趋势,使用移动平均方法对数据进行平滑处理。移动平均能够有效地过滤短期波动,突出长期趋势。
\subsection{ARIMA模型}
本文使用自回归积分滑动平均模型(ARIMA)对汇率数据进行建模。通过对数据的自相关函数(ACF)和偏自相关函数(PACF)的分析识别数据的自相关结构,预测未来24个月的汇率变化趋势。模型参数通过赤池信息准则(AIC)和贝叶斯信息准则(BIC)进行优化,以确保模型的准确性。使用自动ARIMA函数(auto.arima)选择最佳模型,并进行拟合。通过模型的残差分析,验证模型的有效性和准确性。

\section{波动性分析}
\subsection{GARCH模型}
为了捕捉汇率波动的特征,本文使用广义自回归条件异方差模型(GARCH)进行建模。通过拟合GARCH(1,1)模型,分析汇率的波动性,并预测未来的波动趋势。模型拟合后,通过AIC、BIC等信息准则评估模型的优劣。
\subsection{EGARCH模型}
除此之外,本文还额外采用了指数广义自回归条件异方差模型(EGRACH)。GARCH模型用于捕捉时间序列数据中的波动聚集效应,而EGARCH模型则进一步考虑了波动性的不对称性。通过这两种模型,可以更精确地预测未来汇率的波动情况,并分析其背后的驱动因素。

\section{风险分析}
\subsection{VaR分析}
在评估汇率波动带来的潜在风险时,本文采用了VaR(风险价值)方法。VaR方法能够量化特定置信水平下的最大潜在损失。本文计算了不同置信水平(如95\%和99\%)下的VaR值,以评估汇率波动对投资组合可能造成的影响。
\subsection{情景分析和压力测试}
此外,本文还进行了情景分析和压力测试,模拟不同市场条件下的汇率变化情景,评估极端市场波动对投资的影响。

\section{影响因素分析}
\subsection{宏观经济变量分析}
选取与汇率密切相关的宏观经济变量,如中日两国的GDP增长率、通货膨胀率、利率差异等,分析这些变量与汇率波动的相关性。
\subsection{事件分析}
对特定时间段内发生的重大经济事件,如中美贸易摩擦、日本的量化宽松政策等,分析这些事件对人民币兑日元汇率的影响。
\subsection{政策影响分析}
收集中日两国的外汇政策、货币政策等信息,分析政策变化对汇率的影响。

通过以上方法,本文力求全面、系统地分析人民币兑日元汇率的走势及其影响因素,为未来的汇率预测和风险管理提供科学依据。