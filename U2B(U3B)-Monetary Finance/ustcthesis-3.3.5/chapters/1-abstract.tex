% !TeX root = ../main.tex

\ustcsetup{
  keywords  = {人民币兑日元汇率, 时间序列分析, ARIMA模型, GARCH模型, 汇率波动性, 影响因素},
  keywords* = {CNY/JPY exchange rate, time series analysis, ARIMA model, GARCH model, exchange rate volatility, influencing factors},
}

\begin{abstract}
  本文以人民币兑日元汇率为研究对象,通过收集其2009年至2024年的历史数据,利用时间序列分析和波动性分析等统计工具,对其走势、原因及影响因素进行了全面分析。
  
  首先,通过时间序列图和移动平均线展示了人民币兑日元汇率在过去15年的波动情况,发现其整体呈现出波动性较大的趋势,并且在特定时间段内出现显著的波动。

  接着,本文运用ARIMA模型和GARCH模型对汇率数据进行了建模和预测。ARIMA模型的预测结果显示,汇率在未来24个月内将持续保持下行趋势,但波动幅度较小。GARCH模型的分析结果进一步揭示了汇率的波动性特征,预测未来一段时间内的波动率将有所回升但整体趋于平稳。

  在对影响人民币兑日元汇率的因素分析中,本文发现全球经济状况、中日两国的货币政策、国际贸易环境等因素对汇率波动具有显著影响。例如,中美贸易摩擦、疫情对全球经济的冲击,以及日本的超宽松货币政策等,均对人民币兑日元汇率产生了重要影响。

  本文通过详细的数据分析和模型预测,为理解人民币兑日元汇率的走势及其背后的影响因素提供了科学依据。研究结果显示,尽管汇率未来可能会出现短期波动,但长期来看,其走势将受到多重宏观经济因素的综合影响。对于投资者和政策制定者而言,及时关注这些影响因素,并灵活应对市场变化,至关重要。
\end{abstract}

\begin{abstract*}
  This paper focuses on the exchange rate between the Chinese Yuan (CNY) and the Japanese Yen (JPY), analyzing its trends, causes, and influencing factors using historical data from 2009 to 2024. Utilizing statistical tools such as time series analysis and volatility analysis, the study provides a comprehensive examination of the exchange rate's behavior. 

  First, the paper presents the fluctuations of the CNY/JPY exchange rate over the past 15 years through time series graphs and moving averages, revealing a general trend of high volatility with significant fluctuations during specific periods. 
  
  Next, the ARIMA model and the GARCH model are employed to model and forecast the exchange rate data. The ARIMA model's predictions indicate that the exchange rate will continue to decline over the next 24 months, albeit with smaller fluctuations. The GARCH model's analysis further highlights the volatility characteristics of the exchange rate, predicting a slight increase in volatility in the near future, but an overall trend towards stability.
  
  In the analysis of factors influencing the CNY/JPY exchange rate, the study identifies significant impacts from global economic conditions, the monetary policies of China and Japan, and the international trade environment. For instance, factors such as the China-US trade tensions, the global economic impact of the pandemic, and Japan's ultra-loose monetary policy have all significantly affected the CNY/JPY exchange rate.
  
  Through detailed data analysis and model predictions, this paper provides a scientific basis for understanding the trends of the CNY/JPY exchange rate and the factors behind it. The results indicate that while short-term fluctuations may occur, the long-term trend will be influenced by a combination of multiple macroeconomic factors. For investors and policymakers, it is crucial to stay updated on these influencing factors and respond flexibly to market changes.
  
\end{abstract*}
