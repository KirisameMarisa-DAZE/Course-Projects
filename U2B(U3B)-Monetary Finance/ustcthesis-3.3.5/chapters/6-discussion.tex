% !TeX root = ../main.tex

\chapter{讨论}
\section{影响因素分析}
在对人民币兑日元汇率的影响因素分析中,我们选取了多个宏观经济变量和重大事件进行考察。
\section{宏观经济变量}
通过对中日两国的GDP增长率、通货膨胀率和利率差异等宏观经济变量的分析,我们发现这些变量与人民币兑日元汇率波动之间存在显著的相关性。具体而言,当中国的GDP增长率相对较高时,人民币兑日元汇率往往表现出升值趋势;相反,当日本实施量化宽松政策、导致日元贬值时,人民币兑日元汇率则表现为相对升值。这表明,宏观经济基本面的变化是影响汇率波动的重要因素之一。
\section{重大经济事件}
在研究期间,中美贸易摩擦、日本的经济政策调整等重大事件对汇率波动产生了显著影响。例如,中美贸易摩擦加剧时,市场避险情绪上升,日元作为避险货币升值,导致人民币兑日元汇率下跌。类似的,日本实施大规模经济刺激政策时,日元贬值也对汇率产生了明显的影响。
\section{政策影响}
通过对中日两国外汇政策和货币政策的分析,我们发现政策变化对汇率波动有重要影响。例如,中国央行的货币政策调整、外汇市场的干预措施等,都可能导致人民币兑日元汇率的短期波动。此外,日本央行的货币政策变动也直接影响日元汇率,进而影响人民币兑日元汇率。
