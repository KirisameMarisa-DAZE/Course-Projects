% !TeX root = ../main.tex

\chapter{引言}

汇率作为反映一国货币与另一国货币之间的比价,是国际金融市场中的重要指标。人民币兑日元汇率的波动,不仅影响中日两国之间的贸易和投资,还对全球经济形势产生重要影响。随着全球经济一体化进程的不断深入,理解和预测汇率的走势对于企业决策、投资规划以及政策制定具有重要的现实意义。

人民币兑日元汇率的变化受到多种因素的影响,包括宏观经济政策、国际贸易形势、政治事件以及市场情绪等。在过去的15年中,人民币兑日元汇率经历了多次显著的波动。这些波动不仅反映了中日两国经济形势的变化,也反映了国际经济环境的动荡。例如,中美贸易摩擦、全球金融危机、日本的超宽松货币政策等,都对人民币兑日元汇率产生了深远影响。

为了更好地理解人民币兑日元汇率的历史走势及其背后的原因,本研究将通过收集2009年至2024年的汇率历史数据,利用时间序列分析和波动性分析等统计工具,进行系统的分析和预测。具体而言,本文将采用ARIMA模型和GARCH模型,对汇率数据进行建模和预测,分析其未来的走势及波动性特征。同时,通过探讨影响汇率波动的主要因素,揭示其背后的经济逻辑和机制。

本文的研究将分为以下几个部分:首先,介绍人民币兑日元汇率的历史走势,并通过图表直观展示其变化趋势;其次,利用ARIMA模型和GARCH模型对汇率进行建模和预测,分析未来的走势和波动性;最后,探讨影响汇率波动的主要因素,分析其对人民币兑日元汇率的具体影响。

通过本研究,我们希望为理解人民币兑日元汇率的波动提供科学依据,并为企业和政策制定者提供有价值的参考,帮助他们更好地应对汇率风险和把握市场机遇。