% !TeX root = ./main.tex

\ustcsetup{
  title              = {人民币兑日元汇率走势及影响因素分析:\\基于时间序列与波动性模型的研究},
  title*             = {Analysis of RMB/JPY Exchange Rate Trends and Influencing Factors: 
                        A Study Based on Time Series and Volatility Models},
  author             = {李佩哲},
  author*            = {Li Peizhe},
  speciality         = {生物科学},
  speciality*        = {Mathematics and Applied Mathematics},
  supervisor         = {吴遵~教授},
  % supervisor*        = {Prof. XXX, Prof. XXX},
  % date               = {2017-05-01},  % 默认为今日
  % professional-type  = {专业学位类型},
  % professional-type* = {Professional degree type},
  department         = {生命科学学院},  % 院系,本科生需要填写
  student-id         = {PB21051049},  % 学号,本科生需要填写
  % secret-level       = {秘密},     % 绝密|机密|秘密|控阅,注释本行则公开
  % secret-level*      = {Secret},  % Top secret | Highly secret | Secret
  % secret-year        = {10},      % 保密/控阅期限
  % reviewer           = true,      % 声明页显示“评审专家签名”
  %
  % 数学字体
  % math-style         = GB,  % 可选:GB, TeX, ISO
  math-font          = xits,  % 可选:stix, xits, libertinus
}


% 加载宏包

% 定理类环境宏包
\usepackage{amsthm}

% 插图
\usepackage{graphicx}

% 三线表
\usepackage{booktabs}

% 跨页表格
\usepackage{longtable}

% 算法
\usepackage[ruled,linesnumbered]{algorithm2e}

% SI 量和单位
\usepackage{siunitx}

% 参考文献使用 BibTeX + natbib 宏包
% 顺序编码制
\usepackage[sort]{natbib}
\bibliographystyle{ustcthesis-numerical}

% 著者-出版年制
% \usepackage{natbib}
% \bibliographystyle{ustcthesis-authoryear}

% 本科生参考文献的著录格式
% \usepackage[sort]{natbib}
% \bibliographystyle{ustcthesis-bachelor}

% 参考文献使用 BibLaTeX 宏包
% \usepackage[style=ustcthesis-numeric]{biblatex}
% \usepackage[bibstyle=ustcthesis-numeric,citestyle=ustcthesis-inline]{biblatex}
% \usepackage[style=ustcthesis-authoryear]{biblatex}
% \usepackage[style=ustcthesis-bachelor]{biblatex}
% 声明 BibLaTeX 的数据库
% \addbibresource{bib/ustc.bib}

% 配置图片的默认目录
\graphicspath{{figures/}}

% 数学命令
\makeatletter
\newcommand\dif{%  % 微分符号
  \mathop{}\!%
  \ifustc@math@style@TeX
    d%
  \else
    \mathrm{d}%
  \fi
}
\makeatother
\newcommand\eu{{\symup{e}}}
\newcommand\iu{{\symup{i}}}

% 用于写文档的命令
\DeclareRobustCommand\cs[1]{\texttt{\char`\\#1}}
\DeclareRobustCommand\env[1]{\texttt{#1}}
\DeclareRobustCommand\pkg[1]{\textsf{#1}}
\DeclareRobustCommand\file[1]{\nolinkurl{#1}}

% hyperref 宏包在最后调用
\usepackage{hyperref}
\usepackage{float}

\usepackage{listings}
\usepackage{xcolor}

\definecolor{background}{rgb}{0.95,0.95,0.92}
\definecolor{string}{rgb}{0.58,0,0.82}
\definecolor{comment}{rgb}{0.25,0.5,0.35}
\definecolor{keyword}{rgb}{0.37,0.05,0.64}

\lstdefinestyle{mystyle}{
    backgroundcolor=\color{background},   % 背景颜色
    commentstyle=\color{comment}\itshape, % 注释样式
    keywordstyle=\color{keyword}\bfseries, % 关键词样式
    numberstyle=\tiny\color{gray},        % 行号样式
    stringstyle=\color{string},           % 字符串样式
    basicstyle=\ttfamily\footnotesize,    % 基本字体样式
    breakatwhitespace=false,              % 自动断行
    breaklines=true,                      % 自动换行
    captionpos=b,                         % 标题位置
    keepspaces=true,                      % 保持空格
    numbers=left,                         % 行号位置
    numbersep=10pt,                       % 行号与代码距离
    showspaces=false,                     % 不显示空格
    showstringspaces=false,               % 不显示字符串中的空格
    showtabs=false,                       % 不显示制表符
    tabsize=2,                            % 制表符宽度
    frame=single,                         % 框架
    rulecolor=\color{black},              % 框架颜色
    title=\lstname,                       % 显示文件名
    morekeywords={*,...}                  % 自定义关键词
}

% 支持代码折叠
\lstdefinelanguage{R}{
    keywords={if, else, repeat, while, function, for, in, next, break},
    otherkeywords={TRUE, FALSE, NULL, Inf, NaN, NA, NA_integer_, NA_real_, NA_complex_, NA_character_},
    morecomment=[l]\#,                 % 行注释
    morestring=[b]",                   % 双引号字符串
    morestring=[b]'                    % 单引号字符串
}

\lstset{style=mystyle, language=R}